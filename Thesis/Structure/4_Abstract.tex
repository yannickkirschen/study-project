%!TEX root = ../Thesis.tex

\pagestyle{empty}

\renewcommand{\abstractname}{Abstract}

\begin{abstract}
Das Ziel in der vorliegenden Arbeit ist es, ein Stellwerk für eine Klemmbaustein-Eisenbahn zu entwickeln, um einen Beitrag zur Steigerung der Authentizität und des Unterhaltungswerts von Modelleisenbahnanlagen zu leisten.
\newline
Um das Stellwerk zu entwickeln, wurde eine Analyse realer Stellwerkstechnik durchgeführt sowie die Steuerung von Modelleisenbahnen betrachtet. Besonderer Fokus wurde hier auf die Gleisfreimeldeanlage als Teil eines Stellwerks, sowie die Fahrstraßenlogik gesetzt. Es wurde festgestellt, dass die Implementierung einer Gleisfreimeldeanlage mit Reed-Kontakten sowie die Anwendung des Spurplanprinzips für die Fahrstraßenbildung am besten für eine Klemmbaustein-Eisenbahn geeignet ist.
\newline
Im Verlauf der Arbeit traten Probleme bei der Implementierung der Kommuni\-kations-Schnittstellen zwischen einzelnen Komponenten auf. Diese Probleme konnten bis zur Abgabe der Arbeit nicht gelöst werden, weshalb nur einzelne Komponenten des Stellwerks funktionsfähig sind.
\newline
\newline
\newline
\textit{The goal of the present work is to develop an interlocking for a terminal block railroad in order to contribute to increasing the authenticity and entertainment value of model railroad layouts.}
\newline
\textit{In order to develop the interlocking, an analysis of real interlocking technology was carried out and the control of model railroads was considered. Particular focus was placed on the vacancy detection system as part of a signal box, as well as the route logic. It was found that the implementation of a vacancy detection system with reed contacts and the application of the track plan principle for route formation is best suited for a terminal block railroad.}
\newline
\textit{In the course of the work, problems arose with the implementation of the communication interfaces between individual components. These problems could not be solved by the time the work was submitted, which is why only individual components of the interlocking are functional.}
\end{abstract}
