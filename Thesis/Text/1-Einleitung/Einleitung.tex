\chapter{Einleitung}\label{text:Einleitung}

In der Welt der Modelleisenbahnen findet eine stetige Evolution statt. Was einst als einfaches Hobby mit handbetriebenen Lokomotiven und Weichen begann, hat sich zu einem hochtechnologischen Feld entwickelt, in dem Automatisierung und realitätsnahe Steuerung an vorderster Front stehen. Diese Studienarbeit nimmt sich dieser Entwicklung an und konzentriert sich auf die Automatisierung von Modelleisenbahnen durch die Implementierung einer Gleisfreimeldeanlage. Spezifischer wird das Beispiel einer Klemmbaustein-Eisenbahn untersucht, was dem Projekt eine besondere Praxisnähe und Innovationskraft verleiht. 

\section{Motivation}\label{text:Einleitung:Motivation}

Die Motivation für diese Arbeit liegt in der Herausforderung, die komplexe und vielschichtige Technik der realen Eisenbahnsicherungssysteme in den deutlich kleineren Maßstab der Modelleisenbahnen zu übertragen. Hierbei steht nicht nur die technische Realisierbarkeit im Vordergrund, sondern auch die Frage, wie eine solche Automatisierung die Interaktion mit dem Hobby beeinflusst und bereichert. Ziel ist es, durch die Entwicklung und Implementierung einer funktionierenden Gleisfreimeldeanlage einen Beitrag zur Steigerung der Authentizität, Sicherheit und des Unterhaltungswerts von Modelleisenbahnanlagen zu leisten.

\section{Zielsetzung}\label{text:Einleitung:Zielsetzung}

Die Zielsetzung dieser Studienarbeit ist es, ein hochgradig funktionales Stellwerk inklusive einer Gleisfreimeldeanlage für eine Modelleisenbahn zu entwickeln und zu implementieren. Dabei streben wir an, die komplexen Mechanismen und Prozesse, die in realen Stellwerken und Gleisfreimeldeanlagen zum Einsatz kommen, adaptiv auf das Modell einer Klemmbaustein-Eisenbahn zu übertragen. Die Implementierung soll in der Programmiersprache C erfolgen, um eine möglichst hohe Flexibilität und Effizienz zu gewährleisten.
\newline

Die konkreten Ziele dieser Arbeit sind:

\begin{itemize}
    \item Entwicklung einer Gleisfreimeldeanlage für eine Klemmbaustein-Eisenbahn
    \item Entwicklung eines Stellwerks für eine Klemmbaustein-Eisenbahn
    \item Implementierung der Gleisfreimeldeanlage und des Stellwerks in der Programmiersprache C
    \item Test und Auswertung der Funktionalität der entwickelten Komponenten
\end{itemize}

\section{Aufbau der Arbeit}\label{text:Einleitung:Aufbau-der-Arbeit}

Die vorliegende Arbeit gliedert sich in sechs weitere Kapitel. In \autoref{text:Grundlagen} werden zunächst die Grundlagen erarbeitet, die für das Verständnis der Automatisierung von Modelleisenbahnen notwendig sind. Hierbei werden die Stellwerkstechnik, die Gleisfreimeldung und die Steuerung von Modelleisenbahnen betrachtet. Darauf aufbauend wird in \autoref{text:Methodik} die Methodik der Automatisierung von Modelleisenbahnen erläutert. Hierbei wird insbesondere auf die verschiedenen Möglichkeiten der Kommunikation zwischen den einzelnen Komponenten und die möglichen Sensoren zur Implementierung von Achszählern eingegangen. Im Anschluss werden in \autoref{text:Entwicklung-der-GFA} und \autoref{text:Entwicklung-des-Stellwerks} die nähere Entwicklung und Implementierung der Gleisfreimeldeanlage und des Stellwerks beschrieben. Alle Tests, welche in den vorherigen Kapiteln durchgeführt wurden, werden in \autoref{text:Auswertung} final ausgewertet und diskutiert. Abschließend wird in \autoref{text:Fazit-und-Ausblick} das Projekt reflektiert, ein Fazit gezogen und ein Ausblick auf mögliche Weiterentwicklungen gegeben.

