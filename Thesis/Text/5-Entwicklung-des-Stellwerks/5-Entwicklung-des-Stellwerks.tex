\chapter{Entwicklung des Stellwerks}\label{text:Entwicklung-des-Stellwerks}

In den vorangegangen Kapiteln wurde beschrieben, wie die Stellwerkstechnik bei realen Eisenbahnen funktioniert. Das Wissen über Gleisfreimeldeanlagen wurde auf den Kontext einer Klemmbausteineisenbahn übertragen und eine für diesen Zweck geeignete Gleisfreimeldeanlage entwickelt. In diesem Kapitel soll die Entwicklung des Stellwerks beschrieben werden.

Hierfür wird in \autoref{text:Entwicklung-des-Stellwerks:Fahrstrassenlogik} \nameref{text:Entwicklung-des-Stellwerks:Fahrstrassenlogik} auf den Kern des Stellwerks eingegangen, die Bildung und Auflösung von Fahrstraßen. Anschließend erläutert \autoref{text:Entwicklung-des-Stellwerks:Signaldecoder} \nameref{text:Entwicklung-des-Stellwerks:Signaldecoder} die Ansteuerung der Signale und \autoref{text:Entwicklung-des-Stellwerks:Weichendecoder} \nameref{text:Entwicklung-des-Stellwerks:Weichendecoder} die Ansteuerung der Weichen. Abschließend zeigt \autoref{text:Entwicklung-des-Stellwerks:Bedienung} \nameref{text:Entwicklung-des-Stellwerks:Bedienung} wie das Stellwerk zu bedienen ist.

Das gesamte Kapitel bezieht sich im speziellen auf den in \autoref{abb:Entwicklung-des-Stellwerks:Bahnhof} dargestellten Gleisplan.

\begin{figure}[H]
    \centering
    \includegraphics[width=\textwidth]{Assets/Images/5-Entwicklung-des-Stellwerks/Bahnhof.png}
    \caption{Gleisplan des Bahnhofs}\label{abb:Entwicklung-des-Stellwerks:Bahnhof}
\end{figure}

Der Bahnhof und alle anderen Elemente wurden mit Steinen des dänischen Klemmbausteinherstellers LEGO\textsuperscript{\tiny{\textregistered}} gebaut.

\newpage
\section{Fahrstraßenlogik}\label{text:Entwicklung-des-Stellwerks:Fahrstrassenlogik}

Für die Bildung der Fahrstraßen wurde sich für das topologische Prinzip entschieden, da es die größte Flexibilität liefert. Die informationstechnische Abbildung eines topologischen Gleisplans ist ein Graph. Die beiden gängigsten Methoden, einen Graphen programmatisch darzustellen, sind eine Adjazenzliste und eine verkette Liste, wobei in diesem Projekt eine Adjazenzliste verwendet wurde.

In \autoref{abb:Entwicklung-des-Stellwerks:Fahrstrassenlogik:Bahnhof-Graph} ist der Graph des in \autoref{abb:Entwicklung-des-Stellwerks:Bahnhof-Gleisplan} dargestellten Gleisplans zu sehen.

\begin{figure}[H]
    \centering
    \includegraphics[width=\textwidth]{Assets/Images/5-Entwicklung-des-Stellwerks/Bahnhof-Graph.png}
    \caption{Gleisplan des Bahnhofs als Graph}\label{abb:Entwicklung-des-Stellwerks:Fahrstrassenlogik:Bahnhof-Graph}
\end{figure}

Dieser Graph kann dann wie in \autoref{abb:Entwicklung-des-Stellwerks:Fahrstrassenlogik:Bahnhof-Adjazenzliste} in einer Adjazenzliste dargestellt werden:

\begin{figure}[H]
    \centering
    \includegraphics[width=.2\textwidth]{Assets/Images/5-Entwicklung-des-Stellwerks/Bahnhof-Adjazenzliste.png}
    \caption{Gleisplan des Bahnhofs als Adjazenzliste}\label{abb:Entwicklung-des-Stellwerks:Fahrstrassenlogik:Bahnhof-Adjazenzliste}
\end{figure}

Diese Liste, die streng genommen ein assoziatives Array ist, kann in einer einfachen Datenstruktur gespeichert werden. Zur Bildung einer Fahrstraße werden mithilfe der Depth-First-Search alle möglichen Pfade des Graphen gefunden. Anschließend wird der beste Pfad ausgewählt, der dann als Fahrstraße gestellt wird. Im hier verwendeten Beispiel würden für die Fahrstraße \(A \rightarrow P1\) die Pfade

\[A \rightarrow W1 \rightarrow N1 \rightarrow P1\] und
\[A \rightarrow W1 \rightarrow N2 \rightarrow P2 \rightarrow W2 \rightarrow P1\]

gefunden werden. Letzterer ergibt zwar mathematisch Sinn, kann aber aufgrund der Weiche nicht befahren werden. Daher ist eine Filterung notwendig, die schon in der Suche an einer Weiche entscheidet, ob ein Pfad logisch ist oder nicht. In diesem Fall würden für die Fahrstraße folgende Bedingungen gelten:

\[A: Fahrt\]
\[W1: rechts\]
\[N1: Halt\]
\[P1: Halt\]

Die Software prüft nun, ob die benötigten Fahrwegelemente frei sind und stellt sie in die entsprechenden Positionen. Ist die Fahrstraße nicht möglich, weil eine Weiche zum Beispiel Teil einer anderen Fahrstraße ist, wird die Fahrstraße nicht gestellt und ein Fehler ausgegeben. Zum Stellen von Signalen und Weichen werden die Befehle an die jeweiligen Decoder geschickt.


\newpage
\section{Signaldecoder}\label{text:Entwicklung-des-Stellwerks:Signaldecoder}

Wie zuvor bereits erwähnt, wurde der Signaldecoder aufgrund von Problemen mit der Kommunikation zwischen den Komponenten nicht implementiert. Er soll daher hier konzeptuell beschrieben werden.

Der Signaldecoder ist ein vergleichsweise simple Software, die auf einem Raspberry Pi Pico läuft. An diesem sind die LEDs aller Signale direkt angeschlossen (freilich mit einem passenden Widerstand). Initial wird dem Decoder von der zentralen Steuerung die Zuordnung von Signalen und I/O-Pins übermittelt. Danach kann er Befehle zum Stellen der Signale empfangen.

\begin{figure}[H]
    \centering
    \includegraphics[width=.4\textwidth]{Assets/Images/5-Entwicklung-des-Stellwerks/Signal.jpg}
    \caption{Klemmbaustein-Signal}\label{abb:Entwicklung-des-Stellwerks:Signal}
\end{figure}

Sofern die Signale des Stellwerks nur zwei Signalbilder darstellen sollen, also \textit{Halt!} und \textit{Fahrt!}, werden nur zwei LEDs pro Signal benötigt, wobei nur eins von beiden gleichzeitig leuchten kann. In diesem Fall kann durch die Verwendung eines Wechsler-Relais die Anzahl von Signalen, die ein Decoder steuern kann, verdoppelt werden, da nur noch ein I/O-Pin zur Steuerung benötigt wird.


\newpage
\section{Weichendecoder}\label{text:Entwicklung-des-Stellwerks:Weichendecoder}

Aufgrund der zuvor erwähnten Probleme bei der Kommunikation, wurde auch der Weichendecoder nicht implementiert. Hierfür kann eine mögliche Umsetzung konzeptuell beschrieben werden.

Je nachdem welche Weichen verwendet werden, gibt es verschiedene Möglichkeiten sie zu stellen:

\begin{description}
    \item[Integrierte Steuerung] Im einfachsten Fall wird eine Weiche genutzt, die entweder bereits eine integrierte Steuerung hat, oder aber miteiner solchen nachgerüstet werden kann. Weichen mit integrierter Steuerung gibt es beispielsweise von LEGO\textsuperscript{\tiny{\textregistered}} für das 12V- und 9V-System.
    \item[Elektromagnet] Hat die verwendete Weiche keine integrierte Steuerung und lässt sich auch nicht mit dieser nachrüsten, muss eine eigene Lösung entwickelt werden. Diese Weichen sind für den Handbetrieb konzipiert und haben eine Noppe, auf der ein Stein zum vereinfachten Stellen befestigt werden kann. Eine Möglichkeit ist es, einen Permanentmagneten auf dieser Noppe zu platzieren und in geeignetem Abstand einen Elektromagneten anzubringen. Je nach Polung kann die Weiche gestellt werden.
    \item[LED als Indikator] Die simpelste Lösung ist eine LED, die in der Nähe der Weiche angebracht ist und die gewünschte Stellung anzeigt. Der Anwender muss die Weiche dann manuell in die richtige Stellung bringen und beispielsweise über einen Knopf deren Endlage bestätigen.
\end{description}

Eine mögliche Erweiterung der Weichensteuerung ist die Nutzung von elektrischen Kontakten, die die korrekte Endlage einer Weiche überprüfen. Somit würde die Funktionalität einer realen Weiche recht genau nachgebildet werden.


\newpage
\section{Bedienung}\label{text:Entwicklung-des-Stellwerks:Bedienung}

