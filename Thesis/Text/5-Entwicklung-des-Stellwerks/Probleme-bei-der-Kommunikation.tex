\section{Probleme bei der Kommunikation}\label{text:Entwicklung-des-Stellwerks:Probleme-bei-der-Kommunikation}

In \autoref{text:Methodik:Kommunikation} \nameref{text:Methodik:Kommunikation} wurde erläutert, wie die einzelnen Komponenten des Stellwerks miteinander kommunizieren können. Zunächst wurde ein CAN-Bus aufgebaut, der über eine USB-Brücke mit dem Computer über eine serielle Schnittstelle verbunden ist. Hierbei wurde beobachtet, der Bus an sich nur selten funktioniert. Das Fehlerbild beschränkt sich auf ein Volllaufen des Puffers beim Sendern einer Nachricht. Es wurde weiter festgestellt, dass der Fehler nicht deterministisch ist, da in manchen Fällen Daten gesendet und empfangen werden können.
Zu Testzwecken wurde der in \autoref{text:Methodik:Kommunikation:CAN-Bus} \nameref{text:Methodik:Kommunikation:CAN-Bus} dargestellte Testaufbau zweier Raspberry Pi Picos (\autoref{abb:Methodik:Kommunikation:CAN-Bus}) an die USB-Brücke angeschlossen. In seltenen Fällen gelang es, die Kommunikation beider Picos auf dem Computer mitzulesen. In der Regel stoppte der Bus, sobald man die Brücke angeschlossen hat.
Erst dann fiel auf, dass selbst jener Testaufbau nur selten funktioniert. Die Theorie, dass die Kontakte der Breadboards nicht sauber geschlossen werden, konnte weder bestätigt, noch widerlegt werden.

Als Notlösung wurde anschließend versucht, mit UART eine Art Bus zu simulieren (siehe \autoref{text:Methodik:Kommunikation:UART} \nameref{text:Methodik:Kommunikation:UART}). Hierbei kam es zu diversen Problemen, unter anderem dem Senden von Phantom-Bytes und der Kodierung empfangener Daten. Daher musste auch dieser Ansatz verworfen werden.
