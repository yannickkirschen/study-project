\section{Softwaretests}\label{text:Entwicklung-der-GFA:Softwaretests}

Die Softwaretests sind ein wichtiger Bestandteil der Softwareentwicklung. Sie dienen dazu, die Funktionalität der Software zu überprüfen und Fehler zu finden. In diesem Abschnitt werden die Softwaretests für die Gleisfreimeldeanlage beschrieben.\newline
Die Tests für die Gleisfreimeldeanlage sind Software-seitig in zwei Kategorien unterteilt:
\begin{itemize}
    \item Tests für Geraden
    \item Tests für Weichen
\end{itemize}
Die Test-Skripte wurden nach und nach entwickelt und immer wieder ausgeführt. Somit wurden zunächst die Methoden zur Initialisierung der einzelnen Komponenten getestet und später die Methoden zur Überprüfung der Funktionalität der Gleisfreimeldeanlage.

\subsection{Tests für Geraden}\label{text:Entwicklung-der-GFA:Softwaretests:Tests-für-Geraden}

Das Testprogramm startet mit der Initialisierung der einzelnen Komponenten. Da das Skript zu lange ist um es hier komplett darzustellen, wird nur ein Ausschnitt dessen gezeigt. Für das Verständnis ist es wichtig die folgenden Definitionen zu kennen:
\begin{itemize}
    \item p1, p2, p3, p4 sind Kontaktpunkte
    \item d1 ist ein gerichteter Achszähler, der die Kontaktpunkte p1 und p2 verwendet
    \item d2 ist ein gerichteter Achszähler, der die Kontaktpunkte p3 und p4 verwendet
    \item c ist ein Counter/ Streckenabschnitt
    \item v ist die Gleisfreimeldeanlage
\end{itemize}

