\section{Implementierung der Gleisfreimeldeanlage}\label{text:Entwicklung-der-GFA:Implementierung-der-Gleisfreimeldeanlage}

In diesem Abschnitt wird die Implementierung der Gleisfreimeldeanlage mit Schwerpunkt auf den Achszählern beschrieben. Des Weiteren werden alle Tests --- sowohl die Software- als auch die Hardware-Tests --- beschrieben, die durchgeführt wurden, um die Funktionalität der Anlage zu gewährleisten. Hierbei ist zu beachten, dass die Implementierung der Gleisfreimeldeanlage in keinster Weise die Logik einer Route über mehrere Streckenabschnitte hinweg überprüft. Diese Überprüfung findet in der Stellwerkstechnik statt und wird in \autoref{text:Entwicklung-des-Stellwerks:Fahrstrassenlogik} \nameref{text:Entwicklung-des-Stellwerks:Fahrstrassenlogik} beschrieben.

\subsection{Achszähler}\label{text:Entwicklung-der-GFA:Implementierung-der-Gleisfreimeldeanlage:Achszähler}

Die (gerichteten) Achszähler sind der Kern der Gleisfreimeldeanlage. Jeder gerichtete Achszähler besteht hierbei aus zwei Reed-Kontakten (Kontaktpunkte), welche mit einem Mikrocontroller verbunden sind.

\subsubsection{Definitionen der Datenstrukturen}\label{text:Entwicklung-der-GFA:Implementierung-der-Gleisfreimeldeanlage:Achszähler:Definitionen-der-Datenstrukturen}

Die Achszähler sind in der Software als Datenstrukturen definiert, um die Werte der Achszähler zu speichern und zu verarbeiten. Die Definition der Datenstrukturen ist in \autoref{code:Entwicklung-der-GFA:Implementierung-der-Gleisfreimeldeanlage:Achszaehler:Definitionen-der-Datenstrukturen} dargestellt. Ein Kontaktpunkt wird durch das Struct \texttt{rail\_contact\_point\_t} repräsentiert und hat nur eine ID, die den Kontaktpunkt eindeutig identifiziert. Ein gerichteter Achszähler wird durch das Struct \texttt{rail\_contact\_point\_directed\_t} repräsentiert und hat eine ID, zwei Kontaktpunkte (einen inneren und einen äußeren) und zwei boolesche Variablen, die anzeigen, ob ein Zug in den Achszähler einfährt oder ihn verlässt.
\newline
Wenn ein Achszähler von zwei Streckenabschnitten verwendet wird oder die Strecke in beide Richtungen befahren werden kann, werden zwei gerichtete Achszähler verwendet, welche sich die Kontaktpunkte teilen. Somit kommt es nicht zu Konflikten mit den Kontaktpunkten \texttt{inner} und \texttt{outer}.

\begin{margin}
    \begin{lstlisting}[language=C, caption={Definition der Datenstrukturen}, label={code:Entwicklung-der-GFA:Implementierung-der-Gleisfreimeldeanlage:Achszaehler:Definitionen-der-Structs}]
typedef struct {
    int id;
} rail_contact_point_t;

typedef struct {
    int id;

    rail_contact_point_t *outer;
    rail_contact_point_t *inner;

    bool is_entering;
    bool is_leaving;
} rail_contact_point_directed_t;
    \end{lstlisting}
\end{margin}

Ein Streckenabschnitt wird in der Software als Counter (Zähler) repräsentiert. Ein Counter hat eine ID, einen Zählerstand (count), ein Array von gerichteten Achszählern, samt ihrer Anzahl und ein weiteres Array von gerichteten Achszählern um festzustellen, ob ein Zug den Streckenabschnitt verlässt und alle Achsen über den gleichen Achszähler fahren. Verlassen nicht alle Achsen den Streckenabschnitt über den gleichen Achszähler, würde dies bedeuten, dass der vordere Teil des Zuges bei einer Weiche in eine andere Richtung fährt als der hintere Teil, was zur direkten Entgleisung führen kann. Die Definition der Datenstruktur \texttt{rail\_counter\_t} ist in \autoref{lst:Structs2} dargestellt.

\begin{margin}
    \begin{lstlisting}[caption={Definition der Datenstruktur rail\_counter\_t},label={lst:Structs2},language=C]
typedef struct {
    int id;
    int count;

    rail_contact_point_directed_t **contact_points_directed;
    int number_contact_points_directed;

    rail_contact_point_directed_t **directions_to;
    int number_directions_to;
} rail_contact_counter_t;
    \end{lstlisting}
\end{margin}

Mehrere Counter werden in der Software als Gleisfreimeldeanlage (vacancy-detection) zusammengefasst. Eine Gleisfreimeldeanlage hat eine ID und ein Array von Zählern, samt ihrer Anzahl. Ein Objekt vom Typ \texttt{rail\_vacancy\_t} wird in der Software verwendet, um die Gleisfreimeldeanlage zu repräsentieren. Hierbei wird dieses Objekt --- oder Pointer auf dieses Objekt --- an die Funktionen übergeben, um auf die Gleisfreimeldeanlage zuzugreifen. Die Definition der Datenstruktur \texttt{rail\_vacancy\_t} ist in \autoref{lst:Structs3} dargestellt.

\begin{margin}
    \begin{lstlisting}[caption={Definition der Datenstruktur rail\_vacancy\_t},label={lst:Structs3},language=C]
typedef struct {
    rail_contact_counter_t **contact_counters;
    int number_contact_counters;
} rail_vacancy_t;
    \end{lstlisting}
\end{margin}

\subsubsection{Logik und Methoden}\label{text:Entwicklung-der-GFA:Implementierung-der-Gleisfreimeldeanlage:Achszähler:Logik-und-Methoden-der-Achszähler}

Für alle in \autoref{text:Entwicklung-der-GFA:Implementierung-der-Gleisfreimeldeanlage:Achszähler:Definitionen-der-Datenstrukturen} \nameref{text:Entwicklung-der-GFA:Implementierung-der-Gleisfreimeldeanlage:Achszähler:Definitionen-der-Datenstrukturen} definierten Datenstrukturen gibt es Methoden zur Initialisierung, diese sind trivial implementiert und werden hier nicht weiter erläutert. Die erste interessantere Methode ist \texttt{rail\_contact\_counter\_change\_direction}, welche die Richtung eines gerichteten Achszählers ändert. Normalerweise darf ein Zug nur in eine Richtung fahren und nicht in einem Streckenabschnitt wenden. Diese Methode wird genau dann aufgerufen, wenn ein Zug in den Streckenabschnitt einfährt, wendet, und wieder über den selben Achszähler den Streckenabschnitt verlässt. Die Methode \texttt{rail\_contact\_counter\_change\_direction} ist in \autoref{lst:Methoden} dargestellt. Hierbei werden zuerst mehrere Checks durchgeführt um die folgenden Fehler zu vermeiden:
\begin{itemize}
    \item Es ist kein Zug im Streckenabschnitt
    \item Es ist nicht klar in welche Richtung der Zug fährt
    \item Der Streckenabschnitt beinhaltet eine Weiche
\end{itemize}
Wenn alle Checks erfolgreich sind, wird der Wert von \texttt{directions\_to} auf den jeweils anderen Wert gesetzt. Somit wird festgelegt, dass der Zug in die andere Richtung fährt. Aufgrund dieser Implementierung kann ein Zug nicht auf einer Weiche wenden.

\begin{margin}
    \begin{lstlisting}[caption={Methoden der Achszähler},label={lst:Methoden},language=C]
void rail_contact_counter_change_direction(rail_contact_counter_t *counter, error_t *error) {
    if (counter->count == 0) {
        error_add(error, 7, "No train in section");
        return;
    }
    if (counter->number_directions_to == 0) {
        error_add(error, 8, "No direction to change to");
        return;
    }
    if (counter->number_contact_points_directed > 2) {
        error_add(error, 9, "Multiple directions to change to");
        return;
    }

    rail_contact_point_directed_t *d1 = counter->contact_points_directed[0];
    rail_contact_point_directed_t *d2 = counter->contact_points_directed[1];

    if (counter->directions_to[0] == d1) {
        counter->directions_to[0] = d2;
    } else {
        counter->directions_to[0] = d1;
    }
}
    \end{lstlisting}
\end{margin}

Für die Streckenabschnitte ist zusätzlich eine Methode zum zurücksetzen der Zählerstände implementiert. Diese Methode ist zu Testzwecken implementiert, setzt den Zählerstand auf 0 und alle Kontaktpunkte der gerichteten Achszähler zurück.


\subsection{Gleisfreimeldung}\label{text:Entwicklung-der-GFA:Implementierung-der-Gleisfreimeldeanlage:Implementierung-der-Gleisfreimeldeanlage}
Die wichtigste Methode für die Gleisfreimeldeanlage ist die Methode\newline \texttt{rail\_vacancy\_trigger}. Diese Methode wird immer dann aufgerufen, wenn ein Reed-Kontakt ausgelöst wird. Hierfür wird zuerst überprüft, welcher Kontaktpunkt ausgelöst wurde und ob dieser Kontaktpunkt der innere oder äußere eines gerichteten Achszählers ist. Anschließend wird überprüft, ob der Zug in den Streckenabschnitt einfährt oder ihn verlässt. Wenn der Zug in den Streckenabschnitt einfährt und der Kontaktpunkt außen liegt, wird der Zählerstand erhöht. Wenn der Zug den Streckenabschnitt verlässt und der Kontaktpunkt innen liegt, wird der Zählerstand verringert. Die Methode \texttt{rail\_vacancy\_trigger} wäre zu lang um sie hier vollständig darzustellen, kann aber im Quellcode eingesehen werden. Der Link dazu ist im Anhang zu finden.
\newline
Diese simple Logik funktioniert nur, wenn ein Zug bereits im Streckenabschnitt ist und bereits klar ist, über welchen Zähler der Zug den Streckenabschnitt verlässt.
\newline
Fährt ein Zug in den Streckenabschnitt ein, wird der überfahrene Achszähler auf \texttt{is\_entering} gesetzt und dieser Achszähler von den möglichen Richtungen des Streckenabschnitts entfernt. Zusätzlich wird geprüft ob der vorliegende Streckenabschnitt eventuell schon durch einen anderen Zug belegt ist.
\newline
Überfährt die erste Achse des Zugs den Achszähler, welcher am Ende des Streckenabschnitts liegt, wird zunächst geprüft ob dieser Achszähler als mögliche Destination des Zugs gesetzt ist. Ist dies der Fall werden alle anderen Achszähler aus \texttt{directions\_to} entfernt, damit der Zug nur über diesen Achszähler den Streckenabschnitt verlassen kann. Außerdem wird dieser Achszähler auf \texttt{is\_leaving} gesetzt und der Zählerstand des Streckenabschnitts verringert.
