\chapter{Methodik}\label{text:Methodik}

Bei der Automatisierung einer Modelleisenbahn eröffnet sich ein breites Spektrum an Entscheidungsmöglichkeiten, die durch verschiedene Ansätze und Technologien realisiert werden können. Jeder dieser Entscheidungspunkte birgt das Potential, Effizienz, Realitätsnähe und den Unterhaltungswert der Modelleisenbahn wesentlich zu verändern. In diese Kapitel werden verschiedene Methoden und Technologien vorgestellt, die für die Umsetzung der Automatisierung einer Modelleisenbahn in Frage kommen. Es werden spezifische Kriterien und Überlegungen beleuchtet, die bei der Auswahl der geeigneten Umsetzungsmethode berücksichtigt werden müssen. Das Ziel besteht darin, ein tiefgreifendes Verständnis dafür zu entwickeln, wie jede Entscheidung die Konzeption und Funktionalität der automatisierten Modelleisenbahn beeinflusst.

\newpage
\input{Text/3-Methodik/Achszähler.tex}
\newpage
\section{Systemkonfiguration}\label{text:Methodik:Systemkonfiguration}

Die Automatisierung einer Modelleisenbahn stellt ein faszinierendes Unterfangen dar, das eine Brücke zwischen technischer Innovation und nostalgischer Leidenschaft schlägt. Die Auswahl der geeigneten Technik zur Realisierung dieses Ziels erfordert eine gründliche Überlegung verschiedener Ansätze. Dieser Abschnitt widmet sich der Vorstellung von zwei maßgeblichen Varianten, die für die Implementierung des Projekts in Erwägung gezogen werden. Dabei werden nicht nur die technischen Aspekte beleuchtet, sondern auch die praktischen Auswirkungen jeder Variante auf das Gesamtprojekt diskutiert.

\subsection{Zentrale Steuerung}\label{text:Methodik:Systemkonfiguration:Zentrale-Steuerung}

Die Idee einer zentralen Steuerung ist in der Welt der Automatisierung keineswegs neu, findet aber in der Konzeption von Modelleisenbahnsystemen eine besondere Anwendung. Kern dieses Ansatzes ist es, sämtliche Komponenten - von den Weichen über die Signale bis hin zu den Achszählern - unter der Ägide einer zentralen Steuereinheit zu vereinen. Ein Raspberry Pi, bekannt für seine Vielseitigkeit und Leistungsfähigkeit, erweist sich als ideale Wahl für eine solche Steuereinheit. Seine Fähigkeit, zahlreiche Sensoren und Aktuatoren über GPIO-Pins anzusteuern, ermöglicht eine nahtlose Integration und Kontrolle der verschiedenen Systemelemente.
\newline
Ein bedeutender Aspekt der zentralen Steuerung ist die Kommunikation mit dem Benutzer. Der Raspberry Pi kann über eine grafische Benutzeroberfläche (GUI) verfügen, die es dem Benutzer ermöglicht, die Modelleisenbahn zu steuern und zu überwachen. Die GUI kann beispielsweise die Position der Züge auf der Anlage anzeigen, die Weichen und Signale steuern und die Achszähler überwachen. Die zentrale Steuerung bietet somit eine umfassende Kontrolle über die Modelleisenbahn und ermöglicht eine intuitive Interaktion mit dem System.

\subsection{Dezentrale Steuerung}\label{text:Methodik:Systemkonfiguration:Dezentrale-Steuerung}

Während die zentrale Steuerung durch ihre Einheitlichkeit besticht, bietet die dezentrale Steuerung eine Flexibilität, die in komplexen oder sich erweiternden Systemen von unschätzbarem Wert sein kann. Hierbei werden die Achszähler und weitere Systemkomponenten als autonome Einheiten konzipiert, die über individuelle Mikrocontroller gesteuert werden. Diese Mikrocontroller, beispielsweise aus der Arduino- oder ESP-Familie, sind dafür verantwortlich, ihre jeweiligen Bereiche autonom zu verwalten und dabei dennoch eine übergeordnete Koordination durch einen zentralen Computer - das virtuelle Stellwerk - zu ermöglichen.
\newline
Diese Methode brilliert durch ihre modulare Natur. Jede Komponente kann unabhängig entwickelt, getestet und bei Bedarf ersetzt oder erweitert werden. Darüber hinaus erlaubt die dezentrale Architektur eine skalierbare Erweiterung der Anlage, ohne dass die Grundstruktur des Systems grundlegend überdacht werden muss. Allerdings erfordert dieser Ansatz eine sorgfältige Planung der Kommunikationswege zwischen den Mikrocontrollern und dem zentralen Stellwerk, um Effizienz und Zuverlässigkeit zu gewährleisten.

\subsection{Entscheidung}\label{text:Methodik:Systemkonfiguration:Entscheidung}

Nach eingehender Betrachtung der Vor- und Nachteile beider Systeme fiel die Entscheidung auf eine innovative Hybridlösung, die die Stärken beider Ansätze vereint. Diese Lösung sieht vor, dass Achszähler, Weichen und Signale dezentral über Mikrocontroller gesteuert werden, wobei mehrere Sensoren und Aktuatoren an einen einzigen Controller angeschlossen werden können. Ein Beispiel hierfür ist die Nutzung eines Raspberry Pi Pico, der mehrere Achszähler steuert und somit die Effizienz des Systems steigert.
\newline
Die zentrale Steuerung des Stellwerks erfolgt durch ein speziell entwickeltes Programm auf einem Laptop, der eine direkte Verbindung zu den Mikrocontrollern unterhält. Diese Konfiguration optimiert die Flexibilität und Modularität der Anlage und ermöglicht gleichzeitig eine zentrale Überwachung und Steuerung. Durch die Kombination der Sensoren und Aktuatoren an einzelnen Mikrocontrollern können die Kosten effektiv gesenkt werden, während die Modularität und Erweiterbarkeit des Systems erhalten bleiben.
\newline
Die Entscheidung für diese Hybridlösung spiegelt das Bestreben wider, eine Balance zwischen technischer Effizienz, Kostenkontrolle und der Realisierung einer realitätsnahen Modelleisenbahn zu finden. Sie symbolisiert einen innovativen Ansatz in der Welt der Modelleisenbahnautomatisierung, der die Grundlage für zukünftige Entwicklungen und Erweiterungen bildet.
\newline
Die Kommunikation zwischen den Mikrocontrollern untereinander und dem Laptop wird in \autoref{text:Methodik:Kommunikation} näher erläutert.

\newpage
\section{Kommunikation}\label{text:Methodik:Kommunikation}

Die effiziente und zuverlässige Kommunikation zwischen dem Laptop, der als Steuerzentrale dient, und den Mikrocontrollern, welche die Achszähler steuern, ist ein fundamentales Element für die Funktionalität des gesamten Systems. Hierbei kommen verschiedene Schnittstellen und Protokolle in Frage, um eine reibungslose und effektive Übertragung von Steuerbefehlen und Zustandsinformationen zu gewährleisten. Im Folgenden werden die charakteristischen Merkmale und Implementierungsdetails der verschiedenen Kommunikationsmöglichkeiten erörtert, wobei auch auf deren Eignung für spezifische Anwendungsfälle eingegangen wird.

\subsection{CAN-Bus}\label{text:Methodik:Kommunikation:CAN-Bus}

Der CAN-Bus, ein für die Automobilbranche konzipiertes serielles Bussystem, ermöglicht eine effiziente Kommunikation zwischen den diversen Steuergeräten innerhalb eines Fahrzeugs. Als Zweidrahtsystem konstruiert, basiert es auf einem High- und einem Low-Pin für den Datenaustausch. Diese Architektur gewährleistet eine robuste Vernetzung der Steuergeräte, wodurch sie Informationen teilen und aufeinander abstimmen können. Hervorzuheben ist die ausgeprägte Fehlertoleranz des Systems sowie seine Fähigkeit, hohe Datenübertragungsraten zu unterstützen. Neben seiner Anwendung in Fahrzeugen eignet sich der CAN-Bus auch hervorragend für Modelleisenbahnen, dank seiner hohen Leistungsfähigkeit und Zuverlässigkeit. Ein CAN-Bus zwischen mehreren Raspberry Pi Pico Mikrocontrollern kann wie folgt realisiert werden:

\begin{figure}[H]
    \centering
    \includegraphics[width=0.7\textwidth]{Assets/Images/3-Methodik/CANBus-Test.png}
    \caption{CAN-Bus zwischen mehreren Raspberry Pi Pico Mikrocontrollern}\label{fig:Methodik:Kommunikation:CAN-Bus}
\end{figure}

In \autoref{fig:Methodik:Kommunikation:CAN-Bus} ist ein CAN-Bus zwischen zwei Raspberry Pi Pico Mikrocontrollern dargestellt. Der CAN-Bus besteht aus einem Kreis, welcher mit 120 Ohm Widerständen an beiden Enden abgeschlossen ist. Die beiden Mikrocontroller sind über den CAN-Bus miteinander verbunden und können so miteinander kommunizieren. Die Kommunikation erfolgt über die Pins GPIO0 und GPIO1 beider Controller. Die Kommunikation zwischen den Mikrocontrollern wird mit der \lstinline{can2040} \footnote{\url{https://github.com/KevinOConnor/can2040}} Bibliothek realisiert. Diese Bibliothek ermöglicht die Kommunikation über den CAN-Bus und bietet eine einfache API, um Nachrichten zu senden und zu empfangen. Die LEDs auf dem Breadboard veranschaulichen das Ping-Pong-Programm, welches die Kommunikation zwischen den beiden Mikrocontrollern darstellt. Der Mikrocontroller 1 sendet eine Nachricht an den Mikrocontroller 2, welcher diese Nachricht empfängt und eine Antwort zurücksendet. Der Mikrocontroller 1 empfängt diese Antwort und sendet wieder eine Nachricht an den Mikrocontroller 2. Dieser Vorgang wiederholt sich in einer Endlosschleife.

\subsection{UART}\label{text:Methodik:Kommunikation:UART}

UART (Universal Asynchronous Receiver Transmitter) ist ein asynchrones seriell arbeitendes Kommunikationsprotokoll, das in der Regel für die Kommunikation zwischen Mikrocontrollern und anderen Geräten verwendet wird. Es ermöglicht die Übertragung von Daten zwischen zwei Geräten über eine einzige Datenleitung. Die Kommunikation erfolgt über die Pins TX und RX beider Geräte.
\newline
Der Raspberry Pi Pico verfügt über zwei UART-Ports. Mithilfe dieser Ports kann ein Bus zwischen mehreren Mikrocontrollern realisiert werden, welcher quasi unendlich erweiterbar ist. Die Kommunikation zwischen den Mikrocontrollern mittels UART ist bereits in der pico-sdk Bibliothek implementiert.

\subsection{Analoge Kommunikation}\label{text:Methodik:Kommunikation:Analoge-Kommunikation}

Die Kommunikation zwischen den Mikrocontrollern kann auch über analoge Signale realisiert werden. Hierbei wird ein Signal von einem Mikrocontroller erzeugt und an den anderen Mikrocontroller übertragen. Dieser Mikrocontroller empfängt das Signal und kann daraufhin eine Aktion ausführen. Die Kommunikation erfolgt über die Pins eines Mikrocontrollers. Die Kommunikation mittels analoger Signale ist jedoch sehr langsam und unzuverlässig. Deshalb wird diese Methode nicht weiter betrachtet.

\subsection{Anmerkung}\label{text:Methodik:Kommunikation:Anmerkung}

Die Verbindung zwischen dem Steuerlaptop und den Mikrocontrollern über die USB-Schnittstelle stellt eine zentrale Komponente des Kommunikationssystems dar. Ein Raspberry Pi Pico agiert hierbei als Brücke, indem er als USB-zu-Bus-Interface fungiert. Die \lstinline{tinyusb} \footnote{\url{https://github.com/hathach/tinyusb}} Bibliothek ermöglicht eine effiziente und flexible Kommunikation über diese Schnittstelle, wodurch eine hohe Kompatibilität und einfache Integration in das Gesamtsystem gewährleistet wird. 

