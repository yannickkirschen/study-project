\section{Achszähler}\label{text:Methodik:Achszähler}

Wie bereits in \autoref{text:Grundlagen:Gleisfreimeldung:Achszähler} \nameref{text:Grundlagen:Gleisfreimeldung:Achszähler} erörtert, ist die Verwendung von Elektromagneten die Standardmethode für Achszähler in realen Eisenbahnsystemen. Diese Technik stößt jedoch bei der Implementierung in Klemmbaustein-Eisenbahnen auf Grenzen, da die Kunststoffräder der Züge das Magnetfeld bei der Durchfahrt nicht beeinträchtigen. Trotz dieser Herausforderung lassen sich Achszähler auf alternative Weise für Modellbahnen umsetzen.
\newline
Im weiteren Verlauf dieses Kapitels werden drei innovative Ansätze zur Realisierung von Schienenkontakten vorgestellt. Durch die Kombination von zwei solcher Schienenkontakte lässt sich ein funktionstüchtiger Achszähler konstruieren, der sowohl die Anzahl der Achsen erfasst als auch die Richtung der Zugfahrt zuverlässig bestimmt.

\subsection{IR-Lichtschranke}\label{text:Methodik:Achszähler:Lichtschranke}

Eine effiziente Methode, Achszähler in Modellbahnsystemen zu implementieren, bietet die Nutzung von Infrarot-Lichtschranken. Dieser Ansatz bedient sich einer Infrarot-LED in Kombination mit einem Fototransistor, um die Präsenz von Objekten zu detektieren. Im Normalzustand sendet die Infrarot-LED kontinuierlich Licht aus, das vom Fototransistor empfangen wird, wodurch ein konstanter Stromfluss gewährleistet ist. Sobald jedoch ein Fahrzeug zwischen LED und Fototransistor hindurchfährt und das Lichtsignal unterbricht, wird der Fototransistor nicht länger beleuchtet und unterbricht seinerseits den Stromfluss. Dieser Wechsel im Zustand des Fototransistors dient als Signal, dass ein Objekt - in diesem Kontext eine Achse des Fahrzeugs - die Lichtschranke passiert hat.
\newline
Durch diese Technologie ist es möglich, die Anzahl der Achsen präzise zu erfassen, welche die Lichtschranke überqueren. Zudem erlaubt die hohe Reaktionsgeschwindigkeit und Zuverlässigkeit von Infrarot-Lichtschranken eine exakte Zählung und Analyse der Überfahrten, was für die Steuerung und Überwachung der Modellbahn von entscheidender Bedeutung ist. 

\subsection{Hall-Sensor}\label{text:Methodik:Achszähler:Hall-Sensor}

Eine alternative Technik zur Implementierung von Achszählern in Modellbahnsystemen besteht in der Anwendung von Hall-Sensoren. Diese Sensoren sind für ihre Empfindlichkeit gegenüber Magnetfeldern bekannt und ermöglichen die Detektion von Fahrzeugüberfahrten auf einzigartige Weise. Die Herausforderung, dass die Räder der Modellbahnfahrzeuge aus Kunststoff bestehen und folglich selbst keine Magnetfelder generieren, wird kreativ gelöst. Durch die Integration kleiner Magnete in die Unterseite der Fahrzeuge, wird ein erkennbares Magnetfeld erzeugt, sobald diese Fahrzeuge die Hall-Sensoren passieren.
\newline
Diese Methode bietet nicht nur eine hohe Detektionsgenauigkeit, sondern eröffnet auch neue Möglichkeiten für die Zugkontrolle und -überwachung innerhalb der Modellbahnanlage. Die Hall-Sensoren können diskret entlang der Gleise positioniert werden, um die Anwesenheit von Zügen zu erkennen, ohne die ästhetische Gestaltung der Anlage zu beeinträchtigen. Darüber hinaus erlaubt diese Technologie eine lückenlose Erfassung der Zugbewegungen, was insbesondere bei komplexen Gleislayouts mit mehreren Zügen von unschätzbarem Wert ist. Durch die präzise Erkennung jedes Magneten kann die exakte Anzahl der Achsen eines jeden Zuges ermittelt werden, welcher sich im nächsten Streckenabschnitt befindet.

\subsection{Reed-Kontakt}\label{text:Methodik:Achszähler:Reed-Kontakt}

Die dritte Methode zur Einrichtung von Achszählern in Modellbahnsystemen basiert auf der Nutzung von Reed-Kontakten. Diese bestehen aus zwei dünnen Metallstreifen, die hermetisch in einem kleinen Glasröhrchen versiegelt sind. Das Prinzip hinter ihrer Funktionsweise ist einfach, aber effektiv: Nähert sich ein Magnetfeld dem Reed-Kontakt, ziehen sich die beiden Metallstreifen aufgrund der magnetischen Anziehungskraft an und schließen den Stromkreis, sodass elektrischer Strom fließen kann. Entfernt sich das Magnetfeld wieder, trennen sich die Streifen, unterbrechen den Stromkreis und stoppen den Stromfluss.
\newline
Diese Technik zeichnet sich durch ihre Robustheit und Zuverlässigkeit aus, da Reed-Kontakte aufgrund ihrer hermetischen Versiegelung gegenüber Umwelteinflüssen wie Staub und Feuchtigkeit unempfindlich sind. Um sie in Modellbahnen zur Achszählung einzusetzen, werden kleine Magnete an den Fahrzeugen angebracht. Jedes Mal, wenn ein solcher Magnet einen Reed-Kontakt entlang der Schiene passiert, wird ein Impuls erzeugt, der gezählt werden kann. Dies ermöglicht eine präzise Überwachung und Steuerung der Züge auf der Anlage.
\newline
Die Verwendung von Reed-Kontakten für Achszähler bietet neben der hohen Zuverlässigkeit den Vorteil, dass sie eine diskrete Integration in die Modellbahnumgebung erlaubt. Ihre einfache Anwendbarkeit und Langlebigkeit machen sie zu einer attraktiven Option für Modellbahnhobbyisten, die eine effiziente und wartungsarme Lösung zur Zugerkennung und -steuerung suchen.

\subsection{Entscheidungsfindung}\label{text:Methodik:Achszähler:Entscheidungsfindung}

Die Auswahl der optimalen Methode zur Implementierung von Achszählern in einem Modellbahnsystem wird durch mehrere Schlüsselfaktoren beeinflusst. In einem privaten Projekt, bei dem das Hauptaugenmerk auf der Minimierung der Kosten liegt, spielen die Anschaffungskosten der jeweiligen Technologie eine entscheidende Rolle. Jedoch endet die Entscheidungsfindung nicht beim Preis; die angestrebte Authentizität und Realitätsnähe der Modellbahnanlage ist ebenfalls ein wichtiger Faktor, der berücksichtigt werden muss.
\newline
Es gilt, einen sorgfältigen Abwägungsprozess zu durchlaufen, bei dem nicht nur die unmittelbaren Ausgaben, sondern auch die langfristige Wartung, die Zuverlässigkeit der Technik und ihr Beitrag zur gesamten Atmosphäre und Funktionalität der Anlage betrachtet werden. So könnte eine kostengünstigere Lösung auf den ersten Blick attraktiv erscheinen, aber möglicherweise Kompromisse in puncto Langlebigkeit oder Realismus erfordern. Umgekehrt könnte eine etwas teurere Option auf lange Sicht durch geringere Wartungsanforderungen und eine verbesserte Gesamtwirkung der Anlage die bessere Investition darstellen.
\newline
Letztlich muss die Entscheidung für eine der Technologien - sei es Infrarot-Lichtschranken, Hall-Sensoren oder Reed-Kontakte - eine ausgewogene Berücksichtigung aller genannten Aspekte beinhalten. Durch eine umfassende Analyse, die sowohl finanzielle Überlegungen als auch den Wunsch nach einer authentischen Modellbahnerfahrung einbezieht, lässt sich die am besten geeignete Methode zur Achszählung identifizieren.
