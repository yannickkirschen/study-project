\section{Stellwerkstechnik}\label{text:Grundlagen:Stellwerkstechnik}

Mit der Eröffnung der ersten Eisenbahnstrecke 1825 in England begann ein neues Zeitalter. Die Eisenbahn ermöglichte es, Personen und Güter schneller und effizienter zu transportieren als je zuvor. Mit zunehmendem Verkehr auf der Schiene stieg auch die Notwendigkeit der Sicherung der Fahrwege. Dieser Abschnitt behandelt die Grundlagen der Stellwerkstechnik, wie sie bei realen Eisenbahnen eingesetzt wird. Dabei wird zunächst der theoretische Hintergrund erläutert. Anschließend werden die drei geläufigsten technischen Umsetzungen vorgestellt. Diese sind mechanische, Relais- und elektronische Stellwerke. Zuletzt wird auf die neueste Entwicklung, das digitale Stellwerk, eingegangen.

\subsection{Grundlagen der Stellwerkstechnik}\label{text:Grundlagen:Stellwerkstechnik:Grundlagen-der-Stellwerkstechnik}

In den Anfangsjahren der Eisenbahn wurde die Notwendigkeit von Sicherungssystemen schnell deutlich. Ein großer Fortschritt war der Einsatz von Telegraphen, um Zugbewegungen zwischen Bahnhöfen zu koordinieren --- ein Verfahren, das heute noch unter dem Namen \textit{Zugleitbetrieb} angewandt wird. Jedoch war dieses Verfahren nicht ausreichend, um die Sicherheit der Fahrwege zu gewährleisten. So kam es immer wieder zu schweren Unfällen mit vielen Toten. Dies führte in der Folge zur Entwicklung elaborierter Sicherungssysteme, die letztlich in die heutige Stellwerkstechnik mündeten. In diesem Abschnitt werden die wichtigsten Konzepte der Stellwerks- und Sicherungstechnik erläutert.

\subsubsection*{Streckenblock}\label{text:Grundlagen:Stellwerkstechnik:Sicherung-des-Schienenverkehrs:Streckenblock}

Das mit Abstand wichtigste Konzept der Sicherungstechnik ist der Streckenblock. Hierbei handelt es sich zunächst um ein theoretisches Konstrukt, das später mit technischen Mitteln angewandt wird. Eine Zugstrecke wird in logische Abschnitte unterteilt, die \textit{Streckenblöcke} oder \textit{Blockabschnitte}. Hierbei gilt die wichtigste Regel der Sicherungstechnik: in einem Streckenblock darf sich immer nur ein Zug zur selben Zeit befinden.

\subsubsection*{Signale}\label{text:Grundlagen:Stellwerkstechnik:Sicherung-des-Schienenverkehrs:Signale}

Ein Signal ist ein technisches Mittel, um den Zugverkehr zu steuern. Konkret sichert es die Einfahrt in einen Streckenblock ab. Signale können in zwei wesentlichen Bauarten ausgeführt sein: als Formsignal und als Lichtsignal. Formsignale sind die ältere Bauart und werden heute nur noch selten eingesetzt. Sie bestehen aus einem Mast, an dem ein oder mehrere bewegliche Arme angebracht sind. Die Stellung der Arme gibt die Bedeutung des Signals an. Lichtsignale sind die heute übliche Bauart. Sie bestehen aus einem Mast, an dem mehrere Lichter angebracht sind. Die Bedeutung des Signals wird durch die Anordnung und Farbe der Lichter angezeigt. In \autoref{abb:Grundlagen:Stellwerkstechnik:Signale} sind beispielhaft ein Form- und ein Lichtsignal dargestellt.

% https://de.wikipedia.org/wiki/Datei:Ks_Signal_NALB.jpg
% https://de.wikipedia.org/wiki/Datei:Formsignale.jpg
\begin{figure}[H]
    \centering
    \subfloat[\centering Formsignale]{{
        \includegraphics[width=.4\textwidth]{Assets/Images/2-Grundlagen/Formsignale.jpg}
    }}
    \qquad
    \subfloat[\centering Lichtsignal]{{
        \includegraphics[width=.4\textwidth]{Assets/Images/2-Grundlagen/Ks-Signal.jpg}
    }}
    \caption{Beispiele für deutsche Eisenbahnsignale}\label{abb:Grundlagen:Stellwerkstechnik:Signale}
\end{figure}

In Deutschland sind mehrere Signalsysteme im Einsatz. Ein Signalsystem beschreibt die Bedeutung der Signale und die zugehörigen Signalbilder. Wie Ampeln im Straßenverkehr können Eisenbahnsignale einem Zug die Weiterfahrt erlauben oder verwehren. Außerdem können zulässige Höchstgeschwindigkeiten und die Richtung, in die ein Fahrweg eingestellt ist, angezeigt werden. Da diese Arbeit ihren Fokus nicht auf Signalisierung legt, soll auf dieses Thema nicht weiter eingegangen werden.

\subsubsection*{Weichen}\label{text:Grundlagen:Stellwerkstechnik:Sicherung-des-Schienenverkehrs:Weichen}

Eine Weiche ist ein technisches Mittel, um die Fahrtrichtung eines Zuges zu ändern. Sie besteht aus mehreren Schienen, die beweglich miteinander verbunden sind. Weichen gehören mit zu den wichtigsten Fahrwegelementen, da erst durch sie ein sinnvoller Eisenbahnbetrieb möglich wird. In \autoref{abb:Grundlagen:Stellwerkstechnik:Weichen} ist beispielhaft eine Weiche dargestellt.

% https://de.wikipedia.org/wiki/Datei:Handweiche_Bayreuth_Sankt_Georgen.JPG
\begin{figure}[H]
    \centering
    \includegraphics[width=.6\textwidth]{Assets/Images/2-Grundlagen/Handweiche.jpg}
    \caption{Beispiel für eine Weiche}\label{abb:Grundlagen:Stellwerkstechnik:Weichen}
\end{figure}

Weiche gibt es in vielen verschiedenen Ausführungen. Die wichtigste Unterscheidung ist die zwischen \textit{Handweichen} und \textit{elektrischen Weichen}. Handweichen werden von Hand umgestellt. Sie sind die ältere Bauart und werden heute nur noch selten eingesetzt. Elektrische Weichen werden von einem Stellwerk aus umgestellt und sind die heute übliche Bauart. Detaillierter Aufbau und Funktion einer Weiche sind für diese Arbeit nicht relevant. Dennoch soll dieses Thema der Vollständigkeit halber erwähnt werden, da das Konzept der Weiche im weiteren Verlauf dieser Arbeit wieder aufgegriffen wird.

\subsubsection*{Flankenschutz}\label{text:Grundlagen:Stellwerkstechnik:Sicherung-des-Schienenverkehrs:Flankenschutz}

Ein weiteres wichtiges Konzept der Sicherungstechnik ist der Flankenschutz. Er verhindert, dass ein Zug mit einem anderen Zug seitlich kollidiert, also in dessen Flanke fährt. Dies kann insbesondere dann auftreten, falls ein Zug eine ungeplante Fahr- oder Rollbewegung durchführt. Ein Beispiel hierfür ist das zu späte Bremsen bei der Einfahrt in einen Bahnhof. Um Flankenschutz zu gewährleisten, werden Weichen, die gar nicht von einem Zug befahren werden, so gestellt, dass ein auf einem anderen Gleis ungeplant fahrender Zug, nicht in die Flanke des anderen Zuges fahren kann. Um dieses Konzept zu verdeutlichen, sei in \autoref{abb:Grundlagen:Stellwerkstechnik:Gleisplan-einfach} ein beispielhafter, vereinfachter Gleisplan dargestellt.

\begin{figure}[H]
    \centering
    \includegraphics[width=\textwidth]{Assets/Images/2-Grundlagen/Gleisplan-Einfacher-Bahnhof-Nur-Gleisplan.png}
    \caption{Beispiel für einen einfachen Bahnhof}\label{abb:Grundlagen:Stellwerkstechnik:Gleisplan-einfach}
\end{figure}

In \autoref{abb:Grundlagen:Stellwerkstechnik:Gleisplan-ohne-Flankenschutz} sei folgendes Szenario dargestellt: Ein Zug fährt von Signal A nach N1 in den Bahnhof ein. Vor P2 steht ein anderer Zug und wartet. Nun kommt dieser Zug unbeabsichtigt ins Rollen, kollidiert mit der Flanke des einfahrenden Zuges und es kommt zu einem Unfall.

\begin{figure}[H]
    \centering
    \includegraphics[width=\textwidth]{Assets/Images/2-Grundlagen/Gleisplan-Einfacher-Bahnhof-Ohne-Flankenschutz.png}
    \caption{Zugfahrt ohne Flankenschutz}\label{abb:Grundlagen:Stellwerkstechnik:Gleisplan-ohne-Flankenschutz}
\end{figure}

Um dies zu verhindern, werden dem Bahnhof zwei Flankenschutzweichen hinzugefügt, wie in \autoref{abb:Grundlagen:Stellwerkstechnik:Gleisplan-mit-Flankenschutz} dargestellt. Nun kann der rollende Zug nicht mehr in die Flanke des einfahrenden Zuges fahren, sondern wird in ein Stumpfgleis geleitet, wo er zum Stehen kommt.

\begin{figure}[H]
    \centering
    \includegraphics[width=\textwidth]{Assets/Images/2-Grundlagen/Gleisplan-Einfacher-Bahnhof-Mit-Flankenschutz.png}
    \caption{Zugfahrt mit Flankenschutz}\label{abb:Grundlagen:Stellwerkstechnik:Gleisplan-mit-Flankenschutz}
\end{figure}

Das Stellwerk stellt sicher, dass der Flankenschutz beim Stellen einer Fahrstraße immer gegeben ist. In der Praxis sind die Gleisanlagen jedoch meist weit umfangreicher und reine Flankenschutzweichen sind selten. Vielmehr sind Flankenschutzweichen Teil von Fahrwegen anderer Züge und nicht immer direkt zu erkennen.

Neben Weichen können auch Signale und andere Fahrwegelemente Flankenschutz bieten. Für diese Arbeit ist jedoch das Verständnis über Flankenschutzweichen ausreichend.

\subsubsection*{Fahrstraße}\label{text:Grundlagen:Stellwerkstechnik:Sicherung-des-Schienenverkehrs:Fahrstrasse}

Der Weg, den ein Zug zurücklegt, setzt sich aus verschiedenen Elementen zusammen:

\begin{itemize}
    \item Gleisabschnitte,
    \item Weichen,
    \item Signale,
    \item weitere Elemente, die hier nicht weiter relevant sind.
\end{itemize}

Die Kombination dieser Elemente wird als \textit{Fahrstraße} bezeichnet. Eine Fahrstraße beginnt und endet in der Regel an einem Signal. Der \ac{Fdl} fordert das Stellwerk zum stellen einer bestimmten Fahrstraße auf, indem er das Start- und Zielsignal eingibt. Aufgabe des Stellwerks ist es nun, einen Fahrweg zwischen den gewünschten Signalen zu finden. Hierbei müssen alle nötigen Fahrwegelemente gefunden und (im Falle von Weichen und Signalen) in die entsprechende Stellung gebracht werden. Wichtig ist hierbei auch das Gewähren von Flankenschutz. Beim Befahren einer Fahrstraße durch einen Zug, werden die Fahrwegelemente nach und nach frei gefahren und können vom Stellwerk wieder für andere Fahrstraßen verwendet werden. Man sagt, die Fahrstraße wird \textit{aufgelöst}.

Die Bildung einer Fahrstraße kann entweder nach dem \textit{Verschlussplanprinzip} oder dem \textit{Spurplanprinzip} geschehen:

\begin{description}
    \item[Verschlussplanprinzip] Alle möglichen Fahrstraßen sind tabellarisch festgehalten und im Stellwerk fest einprogrammiert. Der Verschlussplan enthält für jede Fahrstraße die Ausschlüsse anderer Fahrstraßen und die Stellung der Fahrwegelemente.~\cite[][S. 122 f.]{bib:Sicherung-des-Schienenverkehrs}
    \item[Spurplanprinzp] Die Topologie der Fahrwegelemente mit Nachbarschaftsbeziehungen ist im Spurplan hinterlegt. Um eine Fahrstraße zu bilden wird \textquote{eine Routensuche zwischen Start- und Zielelement entlang der Verbindungen der einzelnen Fahrwegelemente [\dots] durchgeführt.}~\cite[][S. 123]{bib:Sicherung-des-Schienenverkehrs}
\end{description}

Das Verschlussplanprinzip eignet sich vor allem für kleine Stellwerke, während es bei größeren Anlagen für Mensch kaum noch zu handhaben ist. Das Spurplanprinzip ist hingegen sehr flexibel und eignet sich für große Anlagen.~\cite[][S. 124 f.]{bib:Sicherung-des-Schienenverkehrs}

\subsection{Mechanische Stellwerke}\label{text:Grundlagen:Stellwerkstechnik:Mechanische-Stellwerke}

Mechanische Stellwerke sind die historisch älteste Bauart und werden in Deutschland seit Ende des 19. Jahrhunderts im Einsatz. Bis heute machen sie ein Viertel aller bei der Deutschen Bahn in Betrieb befindlichen Stellwerke aus, steuern aber nur rund 15\% des Streckennetzes.~\cite{bib:DB:Stellwerke} Sie sind die einfachste Bauart und werden heute nicht mehr neu gebaut, sondern nur noch instand gehalten. In diesem Abschnitt werden die wichtigsten Konzepte mechanischer Stellwerke erläutert. \autoref{abb:Grundlagen:Stellwerkstechnik:Mechanische-Stellwerke:Hebelbank} zeigt beispielhaft eine Hebelbank in einem mechanischen Stellwerk.

\begin{figure}[H]
    \centering
    \includegraphics[width=.8\textwidth]{Assets/Images/2-Grundlagen/Mechanisches-Stellwerk-Hebelbank.jpg}
    \caption{Hebelbank in einem mechanischen Stellwerk~\cite{bib:stellwerke.de:Einheit}}\label{abb:Grundlagen:Stellwerkstechnik:Mechanische-Stellwerke:Hebelbank}
\end{figure}

Weichen und Signale werden in einem mechanischen Stellwerk über Seilzüge und Hebel gestellt. Die Hebel sind dabei im Stellwerk in einer Hebelbank angebracht, wobei jedem Signal und jeder Weiche je ein Hebel zugeordnet ist. Bei den Signalen handelt es sich im Formsignale, wie im linken Bild in \autoref{abb:Grundlagen:Stellwerkstechnik:Signale} zu sehen ist.
Im Verschlusskasten hinter der Hebelbank, können einzelne Hebel in einem bestimmten System verschlossen werden, sodass ihre Position nicht mehr verändert werden kann. Dazu bedient man sich des Prinzips der Fahrstraßen: in einem mechanischen Stellwerk sind alle möglichen Fahrstraßen fest definiert und über Fahrstraßenhebel wählbar. Mechanische Stellwerke arbeiten also grundsätzlich nach dem Verschlussplanprinzip. Wird ein Fahrstraßenhebel betätigt, so werden wie bei einem Türschloss die für die Fahrstraße benötigten Hebel im Verschlusskasten verschlossen. Ist eine Weiche nicht in der richtigen Stellung, kann der Fahrstraßenhebel nicht betätigt werden.~\cite[][S.173 ff.]{bib:Sicherung-des-Schienenverkehrs}

\begin{figure}[H]
    \centering
    \includegraphics[width=.8\textwidth]{Assets/Images/2-Grundlagen/Mechanisches-Stellwerk-Blocksperren.jpg}
    \caption{Blocksperren in einem mechanischen Stellwerk~\cite{bib:stellwerke.de:Einheit}}\label{abb:Grundlagen:Stellwerkstechnik:Mechanische-Stellwerke:Blocksperren}
\end{figure}

Die mit Abstand wichtigste Sicherheitseinrichtung in einem mechanischen Stellwerk ist das Blockwerk. Hierbei handelt es sich vereinfacht gesagt um eine bidirektionale Schnittstelle zwischen Mechanik und Elektronik. Während ein Stellwerk rein mechanisch betrieben wird, geschieht die Kommunikation mit dem nächsten Stellwerk über Wechselströme. Über das Blockwerk kann sich der \ac{Fdl} beispielsweise vom Nachbarstellwerk die Erlaubnis geben lassen, einen Zug auf die Strecke zu schicken, oder mitteilen, dass ein Zug unterwegs ist. Die Blockfelder fügen sich ebenfalls in die Sperrmechanismen des Verschlusskastens ein.~\cite[][S.179 ff.]{bib:Sicherung-des-Schienenverkehrs}

Aufgrund einer großen Vielfalt von inkompatiblen Stellwerken, wurde in Deutschland 1924 die \textit{Einheitsbauart} eingeführt. Diese vereinheitlichte die Bauweise von mechanischen Stellwerken und ermöglichte so einen einfacheren Betrieb.~\cite[][S.173]{bib:Sicherung-des-Schienenverkehrs}

Weichen können in einem mechanischen Stellwerk auf 800 Meter Entfernung gestellt werden und Signale auf 1800 Meter Entfernung.~\cite{bib:DB:Stellwerke}

\subsection{Relaisstellwerke}\label{text:Grundlagen:Stellwerkstechnik:Relaisstellwerke}

Aufgrund des hohen Personalbedarfs und der geringen Stellentfernung von mechanischen Stellwerken, wurde in den 1930er und 1940er Jahren begonnen, Relais zur Informationsverarbeitung in Stellwerken zu nutzen. Seit Mitte der 1950er Jahre wurden Relaisstellwerke in Deutschland flächendeckend eingesetzt.~\cite[][S. 188]{bib:Sicherung-des-Schienenverkehrs} Relaisstellwerke sind die heute am weitesten verbreitete Bauart und steuern rund 43\% des Streckennetzes.~\cite{bib:DB:Stellwerke} In diesem Abschnitt werden die wichtigsten Konzepte von Relaisstellwerken erläutert. \autoref{abb:Grundlagen:Stellwerkstechnik:Relaisstellwerke:Kf} zeigt beispielhaft eine Stelltafel in einem Relaisstellwerk.

\begin{figure}[H]
    \centering
    \includegraphics[width=\textwidth]{Assets/Images/2-Grundlagen/Kf.jpg}
    \caption{Stelltafel im Stellwerk Köln Hbf (Kf)~\cite{bib:stellwerke.info:Kf}}\label{abb:Grundlagen:Stellwerkstechnik:Relaisstellwerke:Kf}
\end{figure}

Der für den \ac{Fdl} größte Unterschied eines Relaisstellwerks zu einem mechanischen Stellwerks ist die Bedienung: statt Hebeln werden Tasten verwendet und die Gleisanlagen sind schematisch auf einer Stelltafel dargestellt. In großen Stellwerken, wie das des Kölner Hauptbahnhofs in \autoref{abb:Grundlagen:Stellwerkstechnik:Relaisstellwerke:Kf}, findet die Bedienung nicht direkt über die Tafel statt, sondern über Nummernstellpulte. Kleinere Stellwerke haben die Stelltafel in einem Tisch verbaut.

Die Fahrstraßenbildung erfolgt nun über das Drücken zweier Tasten: einem Start und einem Ziel. Da ein Relaisstellwerk nach dem Spurplanprinzip arbeitet, wird die Fahrstraße nach drücken der Tasten automatisch gesucht und gebildet.~\cite[][S. 188 ff.]{bib:Sicherung-des-Schienenverkehrs} \autoref{abb:Grundlagen:Stellwerkstechnik:Relaisstellwerke:Stelltafel-Ausschnitt} zeigt beispielhaft einen Ausschnitt aus einer Stelltafel mit eingestellter Fahrstraße.

% https://de.wikipedia.org/wiki/Datei:Ausschnitt.png
\begin{figure}[H]
    \centering
    \includegraphics[width=\textwidth]{Assets/Images/2-Grundlagen/Stelltafel-Ausschnitt.png}
    \caption{Ausschnitt aus einer beispielhaften Stelltafel~\cite{bib:stellwerke.info:Kf}}\label{abb:Grundlagen:Stellwerkstechnik:Relaisstellwerke:Stelltafel-Ausschnitt}
\end{figure}

Im Technikraum des Stellwerks ist die gesamte Stelllogik über standardisierte Baugruppen realisiert. Verwendet werden spezielle \textit{Signalrelais}, die den hohen Sicherheitsanforderungen genügen. Signale und Weichen werden nun elektrisch über eine Entfernung von bis zu sieben Kilometern gestellt.~\cite{bib:DB:Stellwerke} Die Signale sind als Lichtsignale ausgeführt, wie beispielhaft im rechten Bild in \autoref{abb:Grundlagen:Stellwerkstechnik:Signale} zu sehen ist.

\subsection{Elektronische Stellwerke}\label{text:Grundlagen:Stellwerkstechnik:Elektronische-Stellwerke}

Elektronische Stellwerke sind die jüngste Bauart und steuern rund 32\% des Streckennetzes Der Deutschen Bahn.~\cite{bib:DB:Stellwerke} Die ersten elektronischen Stellwerke wurden Ende der 1970er Jahre in Betrieb genommen. In Deutschland gingen die ersten Stellwerke in den 1980er Jahren ans Netz, flächendeckend jedoch erst ab Anfang der 1990er Jahre. In diesem Abschnitt werden die wichtigsten Konzepte elektronischer Stellwerke erläutert. \autoref{abb:Grundlagen:Stellwerkstechnik:Elektronische-Stellwerke:ESTW} zeigt beispielhaft die Bedienoberfläche eins elektronisches Stellwerks.

% https://de.wikipedia.org/wiki/Datei:ESTW_L_90.PNG
\begin{figure}[H]
    \centering
    \includegraphics[width=\textwidth]{Assets/Images/2-Grundlagen/ESTW-L-90.png}
    \caption{Bedienoberfläche eines elektronischen Stellwerks}\label{abb:Grundlagen:Stellwerkstechnik:Elektronische-Stellwerke:ESTW-L-90}
\end{figure}

Die Bedienung eines elektronischen Stellwerks erfolgt über eine auf einem Computerbildschirm dargestellte Bedienoberfläche, wie in \autoref{abb:Grundlagen:Stellwerkstechnik:Elektronische-Stellwerke:ESTW-L-90} zu sehen ist. Die gesamte Stelllogik wird nicht mehr über Relais, sondern durch Computer realisiert. Die Fahrstraßenbildung erfolgt über einfache Mausklicks. Durch die Verwendung von Computern ist eine einfache Fernsteuerung von Stellwerken möglich, was die Zentralisierung von Stellwerken ermöglicht. Relais- und elektrische Stellwerke haben eine wichtige Architekturentscheidung gemeinsam: die Verkabelung ist sternförmig aufgebaut und die Ansteuerung der Signale und Weichen geschieht über parallele Kupferleitungen. Dadurch liegt die maximale Stellentfernung bei 7 Kilometern.

\subsection{Digitale Stellwerke}\label{text:Grundlagen:Stellwerkstechnik:Digitale-Stellwerke}

Digitale Stellwerke sind eine Sonderbauform von elektronischen Stellwerken und befinden sich in Deutschland noch in der Entwicklungsphase, beziehungsweise sind erst in wenigen Stellwerken im Einsatz.

Der größte Unterschied zu elektronischen Stellwerken ist die Art der Verkabelung: statt sternförmig, ist die Verkabelung in einem digitalen Stellwerk als Ring aufgebaut. Als Medium werden Glasfaserkabel verwendet, und mit diesen ein IP-Netz über den gesamten Stellbereich aufgespannt. Die Ansteuerung der Signale und Weichen geschieht über Ethernet.
