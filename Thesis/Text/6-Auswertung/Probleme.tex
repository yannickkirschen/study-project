\section{Probleme}\label{text:Auswertung:Probleme}

In \autoref{text:Entwicklung-des-Stellwerks:Probleme-bei-der-Kommunikation} wurde auf \nameref{text:Entwicklung-des-Stellwerks:Probleme-bei-der-Kommunikation} zwischen den Modulen des Stellwerks eingegangen. Trotz intensiver Bemühungen konnten diese Herausforderungen bis zum Abschluss der Arbeit nicht zufriedenstellend gelöst werden. Die Schwierigkeiten in der Kommunikation zwischen den Komponenten verhinderten eine erfolgreiche Implementierung sowohl des Signal- als auch des Weichendecoders. Die exakte Ursache der Kommunikationsprobleme blieb unklar, was eine erhebliche Hürde für die weitere Entwicklung darstellte. Aufgrund dieser Problematik war es unmöglich, eine Gesamtprüfung der Stellwerksfunktionalität vorzunehmen, was die Validierung des Systems in seiner beabsichtigten Betriebsumgebung ausschloss. Stattdessen wurden die einzelnen Komponenten des Stellwerks getrennt getestet, um zumindest eine begrenzte Bewertung ihrer Funktionalität zu ermöglichen. Die Ergebnisse dieser Tests werden in den folgenden Abschnitten detailliert beschrieben.
\newline
Die Schwierigkeiten bei der Kommunikation waren allerdings nicht das einzige Hindernis bei der Entwicklung des Stellwerks. Durch die Wahl von Raspberry Pi Picos als Mikrocontroller musste das Projekt in Python oder C umgesetzt werden. Da C die deutlich bessere Performance bietet, wurde diese Sprache gewählt. Allerdings war die Entwicklung in C für das Team eine große Herausforderung, da es für viele Anwendungsfälle weder Erfahrung noch gute Dokumentation anderer Bibliotheken gab. Dies führte zu einer langsameren Entwicklung, Verstädnisproblemen und mehr Fehlern, die behoben werden mussten. Ein weiteres Problem war die mangelnde Erfahrung des Teams mit der Entwicklung von Hardware und die damit verbundenen Schwierigkeiten bei der Fehlersuche. Als Beispiel sei hier die Fehlersuche bei der Kommunikation zwischen den Modulen genannt. Da das Team keine Erfahrung mit der Kommunikation zwischen Mikrocontrollern hatte, war es schwierig, die Ursache des Problems zu finden. Dies führte zu einer längeren Entwicklungszeit und einer geringeren Qualität des Endprodukts.