\section{Gleisfreimeldung}\label{text:Auswertung:Gleisfreimeldung}

Die spezifischen Tests, die in \autoref{text:Entwicklung-der-GFA:Softwaretests:Tests-für-Geraden} \nameref{text:Entwicklung-der-GFA:Softwaretests:Tests-für-Geraden} und \autoref{text:Entwicklung-der-GFA:Softwaretests:Tests-für-Weichen} \nameref{text:Entwicklung-der-GFA:Softwaretests:Tests-für-Weichen} detailliert wurden, demonstrierten erfolgreich die korrekte Funktionsweise der Gleisfreimeldung. Diese Tests, die softwareseitig auf einem Laptop durchgeführt wurden, simulierten die Gleisfreimeldung, wobei die Zustandsänderungen der Kontaktpunkte manuell initiiert wurden. Trotz der Simulationsumgebung reagierte die Software präzise und zuverlässig auf die simulierten Zustandsänderungen, was die Integrität der Gleisfreimeldung unterstreicht.
\newline
Die durchgeführten Testverfahren illustrierten, dass die Gleisfreimeldung in der Lage ist, die Zustände der verschiedenen Gleisabschnitte akkurat zu erfassen. Die Software verarbeitete die eingehenden Signale korrekt und leitete entsprechende Aktionen ein, was die funktionale Tauglichkeit dieses Teils des Systems bestätigt. Diese erfolgreichen Tests sind ein Beweis für die robuste Konzeption und Implementierung der Gleisfreimeldung.
\newline
Die Hardwaretests, die in \autoref{text:Entwicklung-der-GFA:Hardwaretests} \nameref{text:Entwicklung-der-GFA:Hardwaretests} beschrieben wurden, bestätigten die korrekte Interaktion zwischen den Reed-Kontakten und der Software. Die Testszenarien, die auf dem Steckbrett durchgeführt wurden, demonstrierten die korrekte Funktionsweise der Gleisfreimeldung unter realen Bedingungen. Die LEDs, die den Zählerstand und den Fehlercode anzeigen, ermöglichten eine visuelle Überprüfung der Softwarezustände und bestätigten die korrekte Funktionalität der Gleisfreimeldeanlage in der Hardwareumgebung.
\newline
Durch die Kommunikationsprobleme zwischen den Komponenten konnten die Tests der Gleisfreimeldeanlage nicht auf die finale Umgebung - dem Gleisbett einer Klemmbausteineisenbahn - übertragen werden. Dennoch konnten die Tests in einer simulierten Umgebung erfolgreich durchgeführt werden, was die Funktionalität der Gleisfreimeldung bestätigt. Die erfolgreichen Tests der Gleisfreimeldeanlage sind ein Beweis für die Zuverlässigkeit und die Leistungsfähigkeit dieses Teils des Systems.